

\documentclass[usenames,dvipsnames]{beamer}
\usetheme{metropolis}

\usepackage{pgfpages}
%\setbeameroption{show notes on second screen}
\setbeameroption{hide notes}
%\setbeamertemplate{note page}{\insertnote}

\usepackage{amsthm}
\usepackage{amsmath}
\usepackage{amssymb}
\usepackage{amssymb}
\usepackage{pifont} 
\usepackage{esint}
\usepackage{graphicx}
\usepackage{booktabs}
\usepackage{enumitem}
\usepackage{hyperref}
\usepackage{xcolor}
\usepackage{tikz}
\usepackage{listings}
\usetikzlibrary{decorations.pathreplacing}
\usepackage{verbatim}
\usepackage{soul}

\setbeamertemplate{footline}
{
  \hbox{\begin{beamercolorbox}[wd=1\paperwidth,ht=2.25ex,dp=1ex,right]{framenumber}%
      \usebeamerfont{framenumber}\insertframenumber{} / \inserttotalframenumber\hspace*{2ex}
    \end{beamercolorbox}}%
  \vskip0pt%
}


\usepackage{appendixnumberbeamer}
 
%Information to be included in the title page:
\title{\fontsize{15}{0}\selectfont DiD for Big Data in R\\\vspace{0.1cm}\fontsize{12}{30}\selectfont  Theoretical Background}
\author{\fontsize{12}{30}\selectfont  Bradley Setzler, Penn State}
\institute{\fontsize{6}{6}\selectfont \vspace{0.5cm} \textit{This draft compiled on: \today} }
\date{ }
 
 
 \linespread{.8}
 
 
\begin{document}
 
\frame{
\titlepage
}
 
 
 
%\begin{frame}{Advantages of the DiD for Big Data Package in R}
%
%\url{https://github.com/setzler/DiDforBigData}
%
%\begin{itemize}
%\item[$\bullet$]  \textbf{Consistent Estimates:} Even in staggered-DiD designs, it provides the consistent point estimates and valid SEs. 
%\vspace{0.1cm}
%\item[$\bullet$]  \textbf{Transparency:} The simple plug-in formulas provided above are at the high-school level of math. There is no mystery about what the code is doing; it literally takes averages and variances, then adds/subtracts them.
%\vspace{0.1cm}
%\item[$\bullet$] \textbf{User-friendly:} The user only needs to provide the data in a standard data-frame format. It is easy to set the options for event range ($e$), base period ($b$), and  choice of control group ($\mathcal{C}$). It also provides pretty plots automatically.
%\vspace{0.1cm}
%\item[$\bullet$] \textbf{Big-data:} I wrote it natively in a big-data language.
%\begin{itemize}
%\item[-] My package successfully estimates DiD for \textbf{1 million unique individuals in  20 seconds on a single core}. 
%\item[-] The other packages in R require between 1 hour and $\infty$ hours to estimate DiD with 1 million individuals.
%\end{itemize}
%\vspace{0.1cm}
%\item[$\bullet$]  \textbf{Light-weight:} The complete package is less than 0.1 MB in size, so it can easily be transferred onto admin servers, via email, etc. Furthermore, it only has 1 dependency, data.table, which is a standard R package available on nearly any system.
%\end{itemize}
%
%\end{frame}

 

\begin{frame}{Notation and Identification (1/4)}
 
\vspace{-0.1cm}

\textbf{Notation and Definition of Treatment:}

\vspace{-0.05cm}

\begin{itemize}
\item[$\bullet$] $i$: unit of observation. 
\begin{itemize}
\item[$\rightarrow$]  We will often say ``individual".
\end{itemize}

\vspace{0.1cm}
\item[$\bullet$] $t$: calendar time. Sample time frame is $\mathcal{T}$.
\begin{itemize}
\item[$\rightarrow$] We will often say ``year".
\end{itemize}

\vspace{0.1cm}
\item[$\bullet$] $D_{i,t}$: indicator for currently receiving treatment.
\begin{itemize}
\item[$\rightarrow$] Permanent treatment: $D_{i,t} =1 \implies D_{i,t+1}=1,\; t \in \mathcal{T}$.
\end{itemize}

\vspace{0.1cm}
\item[$\bullet$] $G_i \equiv \min\{t: D_{i,t}=1\}$: time that $i$ first receives  treatment.
\begin{itemize}
\item[$\rightarrow$] We will often say ``cohort" or ``onset time".
\item[$\rightarrow$] If $i$ never receives treatment, we can write $G_i = \infty$. 
\item[$\rightarrow$] Note: $D_{i,t} \equiv 1\{t \geq G_i\}$.
\end{itemize}

\vspace{0.1cm}
\item[$\bullet$] $E_{i,t} \equiv (t - G_{i})$: time since first treatment. 
\begin{itemize}
\item[$\rightarrow$] We will often say ``event time".
\item[$\rightarrow$] Sometimes we will consider fixing an event time $e$ years after treatment versus $b$ years before treatment.
\end{itemize}

\vspace{0.1cm}
\item[$\bullet$] Potential outcomes: Let $Y_{i,t}(g)$ denote the outcome that is experienced if treated at $g$.

\vspace{0.1cm}
\item[$\bullet$] Observed outcome: $Y_{i,t} =   Y_{i,t}(\infty) + \sum_{g} 1\{G_i=g\} (Y_{i,t}(g) - Y_{i,t}(\infty))$.
\end{itemize}
 
\vspace{-0.05cm}

\end{frame}


\begin{frame}{Notation and Identification (2/4)}

\textbf{Goal: Identify ATT at an event time.} We assume throughout that the goal is to identify the average treatment effect on the treated (ATT) at event time $e$. Formally, we seek to identify,
\begin{equation} \notag
\textrm{ATT}_{e} \equiv   \mathbb{E}[Y_{i,t}({\color{blue} g}) - Y_{i,t}({\color{red} \infty}) | {\color{blue}E_{i,t}=e}]  
\end{equation}
The identification challenge is that $\mathbb{E}[ Y_{i,t}({\color{red}\infty}) | {\color{blue}E_{i,t}=e}] $ is a counterfactual object -- it is the average outcome that \textit{would have been experienced} by those receiving treatment for $e$ years  \textit{if they had not received treatment}.




\textbf{Decomposition into cohort-specific ATTs.} Define,
\begin{equation} \notag
\begin{aligned}
\textrm{ATT}_{g,e} & \equiv   \mathbb{E}[Y_{i,g+e}({\color{blue} g})- Y_{i,g+e}({\color{red}\infty}) | {\color{blue}G_i = g}]  \\
\omega_{g,e} & \equiv \mathbb{E}[G_i=g | E_{i,t}=e]
\end{aligned}
\end{equation}
where the cohort shares that have been treated at each event time ($\omega_{g,e}$) are observed. Rearranging terms,  
\begin{equation} \notag
\textrm{ATT}_{e} =  \sum_{g} \omega_{g,e} \textrm{ATT}_{g,e}
\end{equation}
Thus, given $e$, it is sufficient to identify  $\textrm{ATT}_{g,e}$, $\forall g \; \text{s.t.} \; \omega_{g,e}>0$.

\end{frame}


\begin{frame}{Notation and Identification (3/4)}

\textbf{Control Group:} Define a control group membership indicator ${\color{red} \mathcal{C}_{g,e}(G_i)}$. At a minimum,   $ {\color{red} \mathcal{C}_{g,e}(G_i)  {=} 1} \implies G_i > (g+e)$. We may further restrict ${\color{red} \mathcal{C}}$ based on context, e.g., some  consider the never-treated control group, ${\color{red} \mathcal{C}_{g,e}(G_i)}  {=}  1\{G_i  {=}  \infty\}$.

\textbf{Assumption 1: Parallel trends.}
\begin{equation} \notag
\begin{aligned}
 \exists b<0 \; \text{s.t.} &\quad \mathbb{E}[Y_{i,g+e}({\color{red} \infty}) - Y_{i,g+b}({\color{red} \infty}) | {\color{red} \mathcal{C}_{g,e}(G_i)  {=} 1}] 
\\ &\quad = \mathbb{E}[Y_{i,g+e}({\color{red} \infty}) - Y_{i,g+b}({\color{red} \infty}) | {\color{blue}G_i = g}], \; \forall e \geq 0
\end{aligned}
\end{equation}
This restricts the relationship between the treated group ${\color{blue}G_i = g}$ and the control group ${\color{red} \mathcal{C}_{g,e}(G_i) {=} 1}$: the change in average outcome for the treated group \textit{would have been the same in the absence of treatment} as that of the control group.

\textbf{Assumption 2: No anticipation.} 
\begin{equation} \notag
\exists b<0 \;\; \text{s.t.} \;\; \mathbb{E}[Y_{i,g+b}({\color{blue} g}) | {\color{blue} G_i = g }] = \mathbb{E}[Y_{i,g+b}({\color{red} \infty}) | {\color{blue} G_i = g }] , \;\; \forall g
\end{equation}
This restricts \textit{when} the treated cohorts respond to treatment. 

\textbf{Note:} Both assumptions need only hold for the event time $e$ and pre-period $b$ chosen by the researcher.
 
\end{frame}



\begin{frame}{Notation and Identification (4/4)}

\textbf{Difference-in-differences:} Define the population estimator,
\begin{equation} \notag 
\text{DiD}_{g,e} {\equiv}  \mathbb{E}[Y_{i,g+e} {-} Y_{i,g+b} | {\color{blue} G_i = g }] {-}
\mathbb{E}[Y_{i,g+e} {-} Y_{i,g+b} | {\color{red} \mathcal{C}_{g,e}(G_i)  {=} 1}]
\end{equation}
It depends only on observed outcomes, not counterfactuals.

\textbf{Impose parallel trends:}  By the parallel-trends assumption, we can replace the second expectation as follows:
\begin{equation} \notag 
\text{DiD}_{g,e} {\equiv}  \mathbb{E}[Y_{i,g+e} {-} Y_{i,g+b} | {\color{blue} G_i = g }] {-}
\mathbb{E}[Y_{i,g+e}({\color{red} \infty}) - Y_{i,g+b}({\color{red} \infty}) | {\color{blue}G_i = g}]
\end{equation}

\textbf{Impose no anticipation:} By the no-anticipation assumption, we can cancel out the two terms involving $Y_{i,g+b}$:
\begin{equation} \notag 
\text{DiD}_{g,e} \equiv  \mathbb{E}[Y_{i,g+e}  | {\color{blue} G_i = g }] -
\mathbb{E}[Y_{i,g+e}({\color{red} \infty})  | {\color{blue}G_i = g}]
=
\text{ATT}_{g,e}
\end{equation}

\vspace{-0.05cm}

Thus, we have proven that $\text{DiD}_{g,e}=\text{ATT}_{g,e}$ if the parallel-trends and no-anticipation assumptions hold for the pair $(g,e)$. 

\textbf{Result:} If  parallel-trends and no-anticipation hold $\forall g \; \text{s.t.} \; \omega_{g,e}>0$, $$\text{ATT}_{e} = \sum_{g:\; w_{g,e}>0} \omega_{g,e} \text{DiD}_{g,e} $$

\vspace{-0.05cm}

\end{frame}


 

\begin{frame}{Estimator used by DiD for Big Data (1/4)}



\textbf{Estimator based on averages.} Replacing population means with sample means, the package  implements the following DiD:

\vspace{-0.55cm}

\begin{equation} \notag  
\textrm{DiD}_{g,e} 
{=}
\underbrace{ \mathbb{E}[Y_{i,g+e} - Y_{i,g+b}  | {\color{blue} G_i {=} g} ] }_{\textbf{Difference for treated group}}
{-} \underbrace{
\left(
  \mathbb{E}[Y_{i,g+e} - Y_{i,g+b} | {\color{red} \mathcal{C}_{g,e}(G_i) {=} 1}]      
\right) 
  }_{\textbf{Difference for control group}}  
\end{equation}
 
\vspace{-0.2cm}

where, following Callaway and Sant'Anna (2021), we require that parallel-trends and no-anticipation holds for one of these 3 possible control groups:
\begin{equation} \notag
\begin{aligned}
\text{``all"} & \quad {\color{red} \mathcal{C}_{g,e}(G_i)} = 
1\{ G_i > (g+e) \} \\
\text{``future-treated"} & \quad {\color{red} \mathcal{C}_{g,e}(G_i)} = 
1\{ G_i > (g+e) \; \& \; G_i < \infty \} \\
\text{``never-treated"} & \quad {\color{red} \mathcal{C}_{g,e}(G_i)} = 
1\{  G_i = \infty \}
\end{aligned}
\end{equation}

\vspace{-0.05cm} 

\textbf{Researcher choices.} The DiD researcher must make 3 choices:
\vspace{-0.1cm}
\begin{itemize}
\item[\textbf{1.}] What is the range of event times $e$  for which you would like $\text{ATT}_e$ estimates? Default: $e=-5,...,5$.
\item[\textbf{2.}] Which pre-period should be the base? Default: $b=-1$.
\item[\textbf{3.}] Which of the 3 control selections ${\color{red} \mathcal{C}}$ to use? Default: ``all".
\end{itemize}


\end{frame}







\begin{frame}{Estimator used by DiD for Big Data (2/4)}

Fix some $(e,b,g)$. Consider testing $\text{ATT}_{g,e} =0$.

\textbf{Notation:} Consider treatment group ${\color{blue}\mathcal{T}} \equiv 1\{G_i =g\}$ and control group ${\color{red}\mathcal{C}}$, which could be any of the three options above.  

Define within-$i$ differences ${\color{blue}A_{i}} \equiv Y_{i,g+e} - Y_{i,g+b}, i\in{\color{blue}\mathcal{T}}$ with mean ${\color{blue}\mu_A}\equiv \mathbb{E}[{\color{blue}A}_{i} | i \in {\color{blue}\mathcal{T}} ]$, and ${\color{red}B_{i}} \equiv Y_{i,g+e} - Y_{i,g+b}, i\in{\color{red}\mathcal{C}}$ with mean ${\color{red}\mu_B} \equiv \mathbb{E}[{\color{red}B_i} |  i\in{\color{red}\mathcal{C}}]$, where subscripts are dropped if unambiguous.

\textbf{Hypothesis Testing:}  Since $\text{ATT}_{g,e} \equiv {\color{blue}\mu_A} - {\color{red}\mu_B}$, consider,
\vspace{-0.03cm}
\begin{equation} \notag
\text{Test statistic:} \;\; \text{DiD}_{g,e} = {\color{blue}\overline{A}} - {\color{red}\overline{B}} , \quad \text{Null} \;\; H_0:   {\color{blue}\mu_A} - {\color{red}\mu_B} = 0
\end{equation}


\vspace{-0.12cm}

\textbf{Central Limit Theorem:} Denote the population variances by ${\color{blue}\sigma^2_A}\equiv \text{Var}[{\color{blue}A}_{i} | i \in {\color{blue}\mathcal{T}} ]$ and ${\color{red}\sigma^2_B} \equiv \text{Var}[{\color{red}B_i} |  i\in{\color{red}\mathcal{C}}]$. By the CLT under the null, and that the samples are drawn independently across $i$,  
\vspace{-0.03cm}
\begin{equation} \notag
 {\color{blue}\overline{A}} \sim_d \mathcal{N}\left( {\color{blue}\mu_A} , \; {\color{blue}\sigma^2_A}/N_A \right), \quad  {\color{red}\overline{B}} \sim_d \mathcal{N}\left( {\color{red}\mu_B} , \; {\color{red}\sigma^2_B}/N_B \right) 
\end{equation}
\begin{equation} \notag
\implies  \text{DiD}_{g,e} = \left(  {\color{blue}\overline{A}} - {\color{red}\overline{B}} \right) \sim_d  \mathcal{N}\left( 0 , \;  {\color{blue}\sigma^2_A}/N_A  + {\color{red}\sigma^2_B}/N_B   \right)
\end{equation}
Thus,  $\text{SE}(\text{DiD}_{g,e}) = \sqrt{{\color{blue}\sigma^2_A}/N_A  + {\color{red}\sigma^2_B}/N_B  }$. The empirical counterpart is trivial to compute (e.g. no matrix inversion needed).

 
\end{frame}









\begin{frame}{Estimator used by DiD for Big Data (3/4)}

\vspace{-0.1cm}

We usually are not interested in the ATT for a specific cohort $g$. Instead, we are usually interested in the ATT that is $e$ years after treatment, averaging across cohorts.

\textbf{Average Effects by Event Time.} Let  $\omega_g$ denote the share of treated units in cohort $g$. Since treated  cohorts are independent of one another at a given $e$, CLT implies,

\vspace{-0.5cm}

\begin{equation} \notag
\text{DiD}_e =   \sum_\mathcal{G} \omega_g \text{DiD}_{g,e},  
\quad
\text{SE}\left( \text{DiD}_e \right) =  
\sqrt{
\sum_{g \in \mathcal{G}} \omega_g^2 \text{Var}(\text{DiD}_{g,e})
}
\end{equation} 

\vspace{-0.25cm}

\textbf{Average across Event Times:} Similarly, we are often interested in an average of  $\text{DiD}_e$ across event times. Letting $\mathcal{E}$ denote a set of event times (e.g. $\mathcal{E} = \{ 1, 2, 3\}$), we are typically interested in the equally-weighted average, which has the following SE:

\vspace{-0.7cm}

\begin{equation} \notag
\text{DiD}_{\mathcal{E}} =  \frac{1}{|\mathcal{E}|} \sum_\mathcal{E} \text{DiD}_e,
\quad
\text{SE}\left( \text{DiD}_\mathcal{E} \right) =  
 \frac{1}{|\mathcal{E}|}
\sqrt{
\sum_{e \in \mathcal{E}} \text{Var}(\text{DiD}_{e})
}
\end{equation} 

\vspace{-0.1cm}

The R package provides all possible  $ \text{DiD}_e$ with their SEs by default, and user-friendly options to estimate various $\text{DiD}_\mathcal{E}$ and their SEs.

\end{frame}







\begin{frame}{Estimator used by DiD for Big Data (4/4)}

 
\textbf{Event study parameters:} For $\mathcal{H} \in \{ \mathcal{T} , \mathcal{C} \}$, one may wish to plot averages across event times:
\begin{equation} \notag
\bar{Y}^\mathcal{H}_{g,e} = \sum_{i \in \mathcal{H}_g}Y_{i,g+e} , \;
\text{SE}(\bar{Y}^\mathcal{H}_{g,e}) = \sqrt{\frac{\sigma^2_{\mathcal{H},g,e}}{|\mathcal{H}_g|}},  \; \sigma^2_{\mathcal{H},g,e} = \text{Var}[Y_{i,g+e} | i \in \mathcal{H}_g]
\end{equation} 
\begin{equation} \notag
\bar{Y}^\mathcal{H}_{e} = \sum_{g \in \mathcal{G}}
\omega_g \bar{Y}^\mathcal{H}_{g,e}
 , \;
\text{SE}(\bar{Y}^\mathcal{H}_{g,e}) = \sqrt{\sum_g \omega_g^2 \text{Var}(\bar{Y}^\mathcal{H}_{g,e})},
\end{equation} 

The package provides these means and SEs by default.

\vspace{0.5cm}

\textbf{Plots:} The package also provides automated plots, both for presenting the event study parameters and for presenting the ATT estimates. These can be plotted by $(g,e)$ or by $e$ (average over $g$).

\end{frame}




\end{document}

